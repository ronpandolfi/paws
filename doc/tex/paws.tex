%% Generated by Sphinx.
\def\sphinxdocclass{report}
\documentclass[letterpaper,10pt,english]{sphinxmanual}
\ifdefined\pdfpxdimen
   \let\sphinxpxdimen\pdfpxdimen\else\newdimen\sphinxpxdimen
\fi \sphinxpxdimen=49336sp\relax

\usepackage[margin=1in,marginparwidth=0.5in]{geometry}
\usepackage[utf8]{inputenc}
\ifdefined\DeclareUnicodeCharacter
  \DeclareUnicodeCharacter{00A0}{\nobreakspace}
\fi
\usepackage{cmap}
\usepackage[T1]{fontenc}
\usepackage{amsmath,amssymb,amstext}
\usepackage{babel}
\usepackage{times}
\usepackage[Bjarne]{fncychap}
\usepackage{longtable}
\usepackage{sphinx}

\usepackage{multirow}
\usepackage{eqparbox}

% Include hyperref last.
\usepackage{hyperref}
% Fix anchor placement for figures with captions.
\usepackage{hypcap}% it must be loaded after hyperref.
% Set up styles of URL: it should be placed after hyperref.
\urlstyle{same}

\addto\captionsenglish{\renewcommand{\figurename}{Fig.}}
\addto\captionsenglish{\renewcommand{\tablename}{Table}}
\addto\captionsenglish{\renewcommand{\literalblockname}{Listing}}

\addto\extrasenglish{\def\pageautorefname{page}}

\setcounter{tocdepth}{1}



\title{paws Documentation}
\date{Feb 21, 2017}
\release{0.4.0}
\author{Lenson A. Pellouchoud}
\newcommand{\sphinxlogo}{}
\renewcommand{\releasename}{Release}
\makeindex

\begin{document}

\maketitle
\sphinxtableofcontents
\phantomsection\label{\detokenize{index::doc}}


Contents:


\chapter{Introduction}
\label{\detokenize{intro:introduction}}\label{\detokenize{intro:paws-the-platform-for-automated-workflows-by-ssrl}}\label{\detokenize{intro::doc}}\label{\detokenize{intro:ch-introduction}}
The \sphinxcode{paws} package aims to provide
a fast and lean platform for building and executing workflows for data processing.
It was originally developed to perform analysis of diffraction images
for research purposes at SLAC/SSRL.
At the core of \sphinxcode{paws} is a workflow engine
that uses a library of operations
to crunch through data and expose select results
while attempting to minimize resource consumption.

\sphinxcode{paws} is currently written in Python,
based on Qt via the PySide bindings.
Internally, \sphinxcode{paws} keeps track of data in Qt-based tree models,
which can be controlled either directly (through the paws api)
or through a gui (employing the Qt model-view framework).

\sphinxcode{paws} also provides an interface to \sphinxcode{xi-cam},
a synchrotron x-ray diffraction data analysis package
written by the CAMERA Institute and
Pandolfi, et al at the Lawrence Berkeley National Lab.

Some the core goals of \sphinxcode{paws}:
\begin{itemize}
\item {} 
Eliminate redundant development efforts

\item {} 
Streamline and standardize routine data analysis

\item {} 
Simplify data storage and provide large-scale analysis

\item {} 
Perform data analysis in real time for results-driven feedback

\end{itemize}

The \sphinxcode{paws} developers would love to hear from you
if you have wisdom, thoughts, haikus, bugs, artwork, or suggestions.
Limericks are also welcome.
Get in touch with us at \sphinxcode{paws-developers@slac.stanford.edu}.


\chapter{Quick Start}
\label{\detokenize{quickstart:sec-quickstart}}\label{\detokenize{quickstart::doc}}\label{\detokenize{quickstart:quick-start}}
Minimal and usually-effective installation instructions.

Here is a reference to the {\hyperref[\detokenize{intro:ch-introduction}]{\sphinxcrossref{\DUrole{std,std-ref}{brief introduction}}}}.

{\hyperref[\detokenize{quickstart:sec-quickstart}]{\sphinxcrossref{\DUrole{std,std-ref}{This chapter}}}} is for setting up \sphinxcode{paws} quickly
in an environment that is prepared to install Python packages with \sphinxcode{pip}.


\chapter{Installation}
\label{\detokenize{installation:ch-installation}}\label{\detokenize{installation::doc}}\label{\detokenize{installation:installation}}
Here are instructions for installing \sphinxcode{paws} from PyPI,
or downloading and testing the \sphinxcode{paws} source code.


\section{Installing with pip}
\label{\detokenize{installation:sec-pip-installation}}\label{\detokenize{installation:installing-with-pip}}
Instructions will go here for installing \sphinxcode{paws} using
the Python package installer \sphinxcode{pip} (currently not implemented).


\section{Downloading Source}
\label{\detokenize{installation:downloading-source}}\label{\detokenize{installation:sec-src-installation}}
The source code for \sphinxcode{paws} is hosted on github.
Clone the repository from \sphinxcode{https://github.com/slaclab/paws.git}.
You should then be able to run \sphinxcode{paws} by invoking
\sphinxcode{python main.py} from the root directory.


\section{Testing}
\label{\detokenize{installation:testing}}\label{\detokenize{installation:sec-testing}}
\sphinxcode{paws} comes with a tests that can be used
to ensure the platform runs as expected.
After {\hyperref[\detokenize{installation:sec-src-installation}]{\sphinxcrossref{\DUrole{std,std-ref}{downloading the source}}}},
invoke \sphinxcode{python -m unittest discover}
from the root directory.


\chapter{API Documentation}
\label{\detokenize{apidoc:api-documentation}}\label{\detokenize{apidoc:ch-apidoc}}\label{\detokenize{apidoc::doc}}
This is the complete auto-generated documentation of the \sphinxcode{paws} package,
made with sphinx-apidoc.


\section{paws package}
\label{\detokenize{apidoc_files/paws:paws-package}}\label{\detokenize{apidoc_files/paws::doc}}

\subsection{Subpackages}
\label{\detokenize{apidoc_files/paws:subpackages}}

\subsubsection{paws.api package}
\label{\detokenize{apidoc_files/paws.api::doc}}\label{\detokenize{apidoc_files/paws.api:paws-api-package}}

\paragraph{Module contents}
\label{\detokenize{apidoc_files/paws.api:module-contents}}\label{\detokenize{apidoc_files/paws.api:module-paws.api}}\index{paws.api (module)}
Module defining the API for paws
\index{core\_app() (in module paws.api)}

\begin{fulllineitems}
\phantomsection\label{\detokenize{apidoc_files/paws.api:paws.api.core_app}}\pysiglinewithargsret{\sphinxcode{paws.api.}\sphinxbfcode{core\_app}}{\emph{app\_args={[}{]}}}{}
Return a reference to a new QCoreApplication or a currently running QApplication.

Input arguments are passed to the QApplication constructor.
If a RuntimeError is thrown,
it is assumed that a QApplication is already running,
and an attempt is made to return a reference to that QApplication.
If that fails, this returns None.
\begin{quote}\begin{description}
\item[{Parameters}] \leavevmode
\sphinxstyleliteralstrong{app\_args} -- arguments to pass to the QApplication constructor

\item[{Returns}] \leavevmode
reference to a new or existing QCoreApplication

\item[{Return type}] \leavevmode
PySide.QtCore.QCoreApplication or None

\end{description}\end{quote}

\end{fulllineitems}

\index{start() (in module paws.api)}

\begin{fulllineitems}
\phantomsection\label{\detokenize{apidoc_files/paws.api:paws.api.start}}\pysiglinewithargsret{\sphinxcode{paws.api.}\sphinxbfcode{start}}{\emph{app\_args={[}{]}}}{}
Instantiate an Operation Manager, a Workflow Manager, and a Plugin Manager. Return references to them.

paws.api.start() calls paws.api.core\_app(),
then sets up and returns references to 
a paws Workflow Manager (paws.api.workflow\_manager),
Operation Manager (paws.api.op\_manager),
and Plugin Manager (paws.api.plugin\_manager).
\begin{quote}\begin{description}
\item[{Parameters}] \leavevmode
\sphinxstyleliteralstrong{app\_args} -- arguments to pass to the QApplication constructor

\item[{Returns}] \leavevmode
references to paws operation manager, workflow manager, and plugin manager

\item[{Return type}] \leavevmode
tuple of paws.core.operations.op\_manager.OpManager, paws.core.workflow.wf\_manager.WfManager, paws.core.plugins.plugin\_manager.PluginManager

\end{description}\end{quote}

\end{fulllineitems}



\subsection{Module contents}
\label{\detokenize{apidoc_files/paws:module-paws}}\label{\detokenize{apidoc_files/paws:module-contents}}\index{paws (module)}

\chapter{Indices and tables}
\label{\detokenize{index:indices-and-tables}}\begin{itemize}
\item {} 
\DUrole{xref,std,std-ref}{genindex}

\item {} 
\DUrole{xref,std,std-ref}{modindex}

\item {} 
\DUrole{xref,std,std-ref}{search}

\end{itemize}


\renewcommand{\indexname}{Python Module Index}
\begin{sphinxtheindex}
\def\bigletter#1{{\Large\sffamily#1}\nopagebreak\vspace{1mm}}
\bigletter{p}
\item {\sphinxstyleindexentry{paws}}\sphinxstyleindexpageref{apidoc_files/paws:\detokenize{module-paws}}
\item {\sphinxstyleindexentry{paws.api}}\sphinxstyleindexpageref{apidoc_files/paws.api:\detokenize{module-paws.api}}
\end{sphinxtheindex}

\renewcommand{\indexname}{Index}
\printindex
\end{document}